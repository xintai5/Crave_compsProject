\documentclass[10pt,twocolumn]{article} 

% use the oxycomps style file
\usepackage{oxycomps}

% read references.bib for the bibtex data
\bibliography{references}


% include metadata in the generated pdf file
\pdfinfo{
    /Title (The Occidental Computer Science Comprehensive Project)
    /Author (Xintai Aaron Ao)
}

% set the title and author information
\title{The Occidental Computer Science Comprehensive Project}
\author{Xintai Ao}
\affiliation{Occidental College}
\email{xao@oxy.edu}
\begin{document}
\maketitle
\section{Problem Context}

For my Occidental Computer Science Comprehensive project, I want to build a web application that can help consumers compare the prices of the "Big Four" food delivery applications: Uber Eats, Postmates, Grubhub, and DoorDash. The goal of this project is to build a fully functioning web application that would eliminate a common problem among people that order food through delivery apps: the hassle of checking multiple delivery apps just to compare their respective pricing for one food order. Essentially, what should attract people to this web application is its ability to save time and money by eliminating the process of having to go through each online delivery app on one's phone just to save a few extra dollars. This web is meant to increase the efficiency in which a user can optimize their delivery experience and maximize their money and eliminate concerns regarding restaurant exclusivity to certain delivery apps. 

For the scope of this project, I will be focusing on restaurants within a 15 mile radius of Occidental College. This range of potential restaurants encapsulates the scope of this project as it will be a group of restaurants that the average person at Occidental College will most likely order from, which includes students and staff alike. According to a study conducted by \citetitle{ResearchGate}, the main demographic of delivery app users looking to save money are students below the age of 20 and are more prone to choose a delivery app that offers more coupons or special-time offers. Though the intended demographic for this project are people with strict budgets or lower incomes such as high school and college students, this web application can still be useful for people that have the general need to save money while ordering food online. The goal of this project is to make food delivery more accessible for people on a budget whilst saving them time and money. 

The most reasonable scope for this comprehensive project would be to build a web application that solely focuses on comparing the food item prices between two major food delivery applications: UberEats and GrubHub. Other delivery applications could also be added into this comparison web application in the future, including DoorDash, Postmates, InstaCart, Yelp, goPuff, and et cetera. Delivery fees and times are listed separately in the resulting lists from the user input. A problem that arises with this comprehensive project is the task of collecting real-time and history data from these major online delivery conglomerates. Though the historical and real-time data from these applications are not readily available to the public, there is a solution to this: web scraping. By web scraping data from these food delivery apps and storing them in a database, I can then accurately determine the differing price points revealing to the user what the best deal would be. This project will not run into any legal issues so long as the intended web application will not be sold for the purpose of monetary gain.

\begin{figure}
    \centering
    \includegraphics[width=.98\linewidth]{COMPS Final paper/Delivery Market.jpeg}
    \caption{
        Saturation of Online Food Delivery Market
    }
    \label{fig:second-page}
\end{figure}

This project will be very useful for the average consumer as it will give them the access to a convenient tool that will save them time and money. This project benefits the average delivery app user as this web app will give them the ability to save money on commonly used service other than through coupons or special limited time deals on the apps. However, this web app has implications to be potentially detrimental for online food delivery businesses; consumers would reap the benefits of saving money on online food deliveries at the expense of these conglomerates. If this project were to be expanded so that a consumer could order and pay without having to change apps, then it could potentially even steer consumers towards specific applications that would offer the best deals, increasing competition amongst delivery apps. However, in the limited time for this project, I will focus on the price comparison component rather than a creating dual functioning food delivery and price comparison web.

\begin{figure}
    \centering
    \includegraphics[width=.98\linewidth]{COMPS Final paper/Covid.jpeg}
    \caption{
        Financial Effect of Covid-19 on Online Food Delivery
    }
    \label{fig:second-page-1}
\end{figure}  

\section{Technical Background}

For this comprehensive project, the overall stack for the web application is going to be: the user enters the food item they want to order into the web application. Some food items the user can choose include "burritos", pad thai", "burgers", and et cetera. Then the app finds and returns two lists of the respective restaurants within the user's general vicinity. Next to each restaurant name will be listed a food item most closely related to the users search item along with its price, delivery time, and delivery fee. If the user enters an invalid search item, rather than a resulting list, a message will pop up explaining their search did not return any results.

For this project, the method in which real-time data will be extracted from UberEats and GrubHub will be through web scraping. As explained in the article \citetitle{WebScraping}, web scraping is the automated retrieval of data from the text of a web with the aid of bots and storing said data in a structured format. 

With the onset of Covid-19 and the subsequent rapid increase in users of online-to-offline food delivery service, the online food delivery market has become largely saturated. According to \citetitle{Saturation}, there have been both major positive and negative impacts on the online delivery market: the positive effect is that the pandemic has provided a suitable condition for online food delivery companies to promote their services. The negative impact that the pandemic has had on the online food delivery market is that it has triggered "the acceleration of market saturation and reduce the market potential". This has led artificially inflated menu pricing from specific restaurants and the addition of hidden fees and costs to both the restaurants and consumers of the online delivery market.

This project aims to prevent consumers from having to deal with these predatory behaviors from major food delivery app companies by finding them the best deals possible. In doing so, the hope for this project is to potentially reinvigorate a largely saturated market. The convenience of ordering food online has ironically become a hassle in itself for a large number of consumers, especially for those still in school or on a budget. 

\section{Prior Work}

Some prior work that has been done to solve similar issues in the online food delivery market are the delivery price comparison apps “Mealme”  and “FoodBoss”.

Mealme is an iOS app that has features similar to what I want in my web application. A breakdown of MealMe: the iOS app first allows the user to set a specific price range for their desired food delivery order. Next, the app will display multiples lists that fit into the consumer's price range including trending items, daily deals, popular restaurants, cheapest deliverable food and drink options, and so forth, all relative the user's location. One problem with this application is that it is only available on iOS. That is why I plan on creating a web application so that it may be available for all users, as long as they have access to a computer or cell phone. I will not try to emulate all the components of Mealme since the problem in doing that would essentially be me trying to recreate an app that hires hundreds of thousands of dollars worth of coders in salary, within four months. That would become a far too complicated task to complete for the given timeline for this project.

The Mealme claims that the app uses “A.I.” to find the best deals possible for its users, which is not true. What the app does do is it web scrapes data from these major delivery company's apps and presents them in a comprehensible way to its users. However, the app’s interface is not fluid, in that it takes a long time to load, and it occasionally does not display a specific delivery service option for a food order. This leaves room for errors such as longer than expected delivery times, missing orders, and inaccurate delivery fee calculations, such as the negligence of hidden service fees unique to each major delivery app.

Foodboss is essentially the Kayak for food delivery services, in that it compares the delivery prices of nearby restaurants. This application's function is the most similar to what I want to do for this project. However, similar to MealMe, the application is only available on the IPhone whereas I want to create a web that is available to all users.

Though this app is most similar to what I want to emulate, for the given timeline and scope of this project, trying to create a web with all the components of MealMe would be a far too daunting and complicated task to complete. However, for the timeline of this project, the most reasonable scope would be to build a web application that solely focuses on comparing the total cost of a set online delivery orders from the four major delivery applications. Delivery fees and taxes will be included into the total cost of each delivery app.

\begin{figure}
    \centering
    \includegraphics[width=.98\linewidth]{COMPS Final paper/Mealme.png}
    \caption{
        MealMe Interface
    }
    \label{fig:second-page-2}
\end{figure}

\section{Methods}

I plan on using Octoparse for this project because this web data extraction tool is free and it also enables users to scrape unstructured data from major applications such as these major food delivery apps. Octoparse can save this data in different formats including through Excel, HTML, and CSV files, which will be necessary for this project. For the purpose of this project, I will be extracting the data into CSV files to then later convert into SQL files, which is more applicable in the case of creating a web app. This data scraping and formatting would otherwise be difficult or near impossible within the given time frame of this project. This is because the extraction of data from these major corporation web applications would be impossible since they are dynamically generated and are formatted based on the cookies of the user logging into the web page. Also, without the legal and written consent of these companies, extracting this data could be illegal. Also, I would need to create my own bots and find a way to illegally extract data from the these major corporation website apps. Though I do not mind ripping off major corporations, for the sake of this project and the amount of time given to complete it, this would be completely impossible and unnecessary.

\begin{figure}
    \centering
    \includegraphics[width=.98\linewidth]{COMPS Final paper/FoodBoss.png}
    \caption{
        FoodBoss Interface
    }
    \label{fig:second-page-3}
\end{figure}

For this comprehensive project, since I plan to utilize web scraping to pull real-time data from the UberEats and GrubHub, the main application stack for this web application project is: How do I want users to interact with this website? 

The goal for this web application at its core is to give its users the ability to aggregate and compare prices between major food delivery apps. The best way to do this is by designing the web application so that it functions similarly as to how a major food delivery app would.

First, the web application would need to have a search bar for specific food items so that the user can search for what they want. In this web application, when a user searches for a restaurant, the order in which the application should work is:

\begin{enumerate}[label=(\Alph*)]
\item The user enters their desired food item.
\item The web app outputs two resulting lists, one for UberEats and one for GrubHub.
\item Each resulting listing list will include the resulting food item name, restaurant name, food item price, delivery fee, and deliery time.
\item If a specific delivery service is not available for that particular restaurant or specific order, then the output on the list will be labeled as "Not Available".
\end{enumerate}

Since UberEats and GrubHub are both dynamically generated and formatted based on the cookies of the user logging into the web page, the method which I will use to scrape each delivery app will have to be tailored to each.

Another method that I can follow in order to better understand and utilize web scraping is by following the \citetitle{Ben} by Ben Minor. This comprehensive tutorial on web scraping uses Python which would be a suitable language for this web application since Python has been the leading web scraping language for the better part of a decade. The tutorial uses the Python BeautifulSoup4 library, which is super lightweight, versatile, and makes quick work of web pages with limited used of JavaScript and animation. Beautiful Soup is a library that sits atop an HTML or XML parser, providing Pythonic idioms for iterating, searching, and modifying the parse tree. This tutorial uses Uber Eats as an example which is perfect for this project. However, it may not be as efficient as using Octoparse as it does a better job in aggregating the process of web scraping multiple websites whereas this method requires extra work for the same result.

\begin{figure}
    \centering
    \includegraphics[width=.98\linewidth]{COMPS Final paper/What is Scraping.jpeg}
    \caption{
        How Web Scraping Works
    }
    \label{fig:second-page-4}
\end{figure}

\section{Evaluation Metrics}

The main goal in testing this project is to evaluate whether or not this web application functions as intended. The important aspects of this web application worth testing is:

\begin{itemize}
    \item Is the app convenient to use? Will have beta testers test that.
    \item Is the web application intuitive?
    \item Is the outputted information correct?
    \item Is the output list load time sub-half second?
    \item Are there any abuse cases? Meaning does it work under the corner cases? 
    \subitem Example cases: Uber Eats doesn't exist; user inputs fake restaurant or items; data feeds become corrupted; i.e. anything in which the processing deviates from normal operations.
    \item How does the web application handle these deviations? How can they be corrected?
\end{itemize}

The best way to evaluate how well this web application project could save money for potential users is through entering a set number of orders from different restaurants into the web application. Each of the orders will be unique in that they each contain different food items or drinks and that they will be from specific restaurants. Some of the order sets will purposefully contain fake items, an inordinate amount of one item, or items not available at said restaurant.

Another way in which this project will evaluated is through the enlisting of random test users. This will include students from high schools surrounding Occidental College, and students, staff members, and even professors from Occidental College. Since these people are the main demographic for this project, only they would be the most suitable testers to earn feedback from. An important aim for this project is the feeling on convenience and lack of frustration for these users. I do not want this web application to feel as if it is an extra unnecessary step for an action as simple as ordering food through an app. I would consider the web application a failure if and only if it fails to save its users money less than 50\% of the time or if it becomes more of an inconvenience than just checking multiple delivery apps for the best deal.

\begin{figure}
    \centering
    \includegraphics[width=.98\linewidth]{COMPS Final paper/TMC.jpeg}
    \caption{
        How Much Extra Customers Pay per App
    }
    \label{fig:second-page-5}
\end{figure}

\section(Results and Discussion)

My web app is successful in finding users the cheapest food items near Occidental College. Through the evaluations of my web app through user testing, I found that users were able to reliably find the cheapest food options in relation to their preferences whilst listing the delivery fees and times associated with each specific order.

Of the ten participants in the evaluation of my web app, nine of them responded with positive reviews, saying they would use this web app in the future to find the cheapest options for food delivery. However, I was unable to properly install a feature where the user could click an element in the resulting list bringing them to the respective food delivery app with their order already selected, which would have been a main motivator in using my web app for these participants.
All of the participants liked the available details in each resulting list of food items in relation to their search which included the food item price and delivery time and fee; the participants all concluded that was a prime motivator in ordering from a specific food delivery app.

In conclusion I believe that my app, and larger project goal, was achieved. I developed a web app that compares the food prices and delivery fees and times of respective popular food delivery apps (UberEats and GrubHub).

However, I believe there is a lot to improve on and I will continue to work on this project in the future as I believe their is a lot of potential for it as there is a lot of interest in an app such this per my survey of numerous Occidental College students. Due to the limited time frame of this project and the complexity of mainstream food delivery apps, such as their implementation of dynamic programming and machine learning which caused the hindrance of my web scrapping, I was not able to fully implement every key feature and food delivery app into this web app.


\section{Ethical Considerations}

According to an article titled \citetitle{Covid}, the impact of the Covid-19 pandemic on the online food delivery market has had both positive and negative consequences: "In the United States, the market has more than doubled during the COVID-19 pandemic, following healthy historical growth of 8 percent". Though the the market for delivery has become popular with the the onset of social distancing, this has also given rise for companies to take advantage of this, leading to potentially ethically ambiguous decisions by the major delivery companies. This includes the artificial inflation of delivery fees and restaurant item costs. This means that ordering a specific item from a restaurant for online delivery is more expensive than if a user were to order in person. The cost of convenience has scaled up online and has been moved into the food delivery market.

This means that delivery fees have been volatile and even differ between each delivery app for the same exact order from a restaurant. If a customer were to order a typical meal from a fast food restaurant say priced around \$25, then the customer total cost (on average) would end up being \$35 as it would also included delivery fees, driver tips, and platform service fees. This is the case for all of the major online food delivery apps in the market. Since customers do not directly see the service commissions that restaurants pay platforms, some restaurants raise their delivery-menu prices to cover this cost, while others opt for pricing consistency, spreading the markup among all customers. Also worth noting is that restaurants themselves receive around only 55 percent of the total customer spend.

According to \citetitle{HiddenFees}, "When you order through a delivery app, you pay multiple parties, including the driver and the companies that offer the apps, like Uber Eats and Postmates. In some cases, you pay the restaurants extra fees as well".
Uber eats mandates a profit-sharing arrangement (Companies cannot opt out of this option unless they have a marketing/business/unique partnership to negate these fees in exchange for something like exclusivity: Chick-fila and DoorDash in Canada: DoorDash offers no kickback if Chickfila exclusively works with DoorDash) where they assert that Uber Eats existing brings revenue to restaurant s they otherwise would not receive, such that they charge restaurants between 15 and 30\% of the order cost and listing fees. Fees in which restaurants simply pass onto the consumers. Fees that increase the 10\% standard service fee that Uber Eats charges on all of their non-membership orders. Generating a cascading fee hierarchy that is passed almost exclusively to the consumer. 

\section{Proposed Timeline}
For the timeline of my project starting this summer, I have to start thinking about programming the project for my testing users. Below is a list of all the dates for my project and major events that need to happen to fulfill finishing my project on time. 



\printbibliography

\end{document}

